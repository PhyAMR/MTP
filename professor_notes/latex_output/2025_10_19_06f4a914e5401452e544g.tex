\documentclass[10pt]{article}
\usepackage[utf8]{inputenc}
\usepackage[T1]{fontenc}
\usepackage{amsmath}
\usepackage{amsfonts}
\usepackage{amssymb}
\usepackage[version=4]{mhchem}
\usepackage{extpfeil}
\usepackage{stmaryrd}
\usepackage{bbold}

\begin{document}
Models of T.P. $30 / 9 / 25$\\
Gaussion Integnals\\
Consider the Goussion distribution (PDF) :

$$
p(x)=c e^{-\frac{a x^{2}}{2}} \quad a>0
$$

$p$ must be nonualized on $(-\infty,+\infty)$ :

$$
\int_{-\infty}^{+} p(x) d x=1 \quad \Rightarrow \quad c \int_{-\infty}^{+\infty} e^{-\frac{a x^{2}}{2}} d x=1
$$

The simplest Gaussion integral •\\
(1) $\quad \int_{-\infty}^{+\infty} e^{-\frac{a x^{2}}{2}} d x=\sqrt{\frac{2 \pi}{a}}$

A more general Goussion integral is


\begin{equation*}
\int_{-\infty}^{+\infty} e^{-\frac{a x^{2}}{2}+b x} d x=? \tag{2}
\end{equation*}


To edcalate this integral we use a charge of vars. The mox of the exponent has charged. Let's find it:

$$
\frac{d}{d x}\left(-\frac{a x^{2}}{2}+b x\right)=-a x+b=0 \Rightarrow x=\frac{b}{a} \quad \text { (b eal) }
$$

We introduce $y=x-\frac{b}{a}$\\
Venify\\
$-\frac{a x^{2}}{2}+b x=-\frac{a}{2}\left(y+\frac{b}{a}\right)^{2}+b\left(y+\frac{b}{a}\right)=\cdots=-\frac{a y^{2}}{2}+\frac{b^{2}}{2 a}$\\
(3) $\int_{-\infty}^{+} e^{-\frac{a x^{2}}{2}+b x} d x=\int_{-\infty}^{+\infty} e^{-\frac{a y^{2}}{2}+\frac{b^{2}}{2 a} d y} \xlongequal{\text { (1) }} \sqrt{\frac{2 \pi}{a}} e^{\frac{b^{2}}{2 a}} \quad \begin{aligned} & a>0 \\ & b \in \mathbb{C}\end{aligned}$ b con also be a complex number.

Let's calculate the integral in (3) when $b=i t \quad t \in \mathbb{R}$. Lat's define

$$
\varphi(t)=\int_{-\infty}^{+\infty} d x e^{i x t} \sqrt{\frac{a}{2 \pi}} e^{-\frac{a x^{2}}{2}}
$$

Note:\\
$\frac{d}{d t} \int f(x, t) d x=\int \frac{\partial f(x, t)}{\partial t} d x$ if $f, \partial_{t} f$ are continuous and $|f(x, t)|<A(x) \quad,\left|\partial_{t} f\right|<B(x), \quad \int A(x) d x<\infty \quad \int B(x) d x<\infty$ ( $f$ is continuous and unifoumly bounded).

$$
\begin{aligned}
& \varphi^{\prime}(t)=\sqrt{\frac{a}{2 \pi} i} \int d x \times e^{i x t} e^{-\frac{a x^{2}}{2}} \\
& =-\frac{i}{\sqrt{2 \pi a}} \int d x e^{i x t} \frac{d}{d x} e^{-\frac{a x^{2}}{2}} \quad \frac{d}{d x} e^{-a x^{2}}=-a x e^{-a x^{2}} \\
& =-\frac{t}{\sqrt{2 \pi a}} \int d x e^{i x t} e^{-\frac{a x^{2}}{2}} \quad(\text { by posts }) \\
& =-\frac{t}{a} \varphi(t) \\
& \Rightarrow \quad \varphi^{\prime}=-\frac{t}{a} \varphi
\end{aligned}
$$

This obifferential $\varphi$. is linear and the solution is $\varphi(t)=c e^{-\frac{t^{2}}{2 a}}$ (check this out!). It must be $\varphi(0)=1 \Rightarrow c=1$

$$
\begin{gathered}
\varphi(t)=e^{-\frac{t^{2}}{2 a}}=e^{\frac{b^{2}}{2 a}} \\
b=i t
\end{gathered}
$$

Characteristic Functions\\
If we are given a PDF $p(x)$, its char. func. is defined as\\
(4) $\varphi(k)=\int e^{i k x} p(x) d x \equiv\left\langle e^{i k x}\right\rangle_{p}$

In general

$$
\langle f\rangle_{p} \equiv \int f(x) p(x) d x
$$

$\varphi(x)$ has a very nice property:

$$
\frac{d \varphi}{d k}=\int(i x) e^{i k x} p(x) d x \rightarrow-\left.i \frac{d \varphi}{d k}\right|_{k=0}=\int x p(x) d x=\langle x\rangle
$$

and in general:\\
(5) $\left.(-i)^{n} \frac{d^{n} \varphi}{d k^{n}}\right|_{k=0}=\int x^{n} p(x) d x=\left\langle x^{n}\right\rangle$\\
wth moment of $P$

$$
n=1,2, \cdots
$$

What is the c.f. Of the Gaussion distribution? Substitute $b \rightarrow$ itc in e. (2), then you get


\begin{equation*}
\int_{-\infty}^{+\infty} e^{-\frac{a x^{2}}{2}+b x} d x=\int e^{-\frac{a x^{2}}{2}+i k x}=\sqrt{\frac{2 \pi}{a}} e^{-\frac{k^{2}}{2 a}} \tag{3}
\end{equation*}


hence the c.f. of the Goussion is $\left(\sqrt{\frac{1}{2 \pi \sigma^{2}}} e^{-\frac{x^{2}}{2 \sigma^{2}}}, a=\frac{1}{\sigma^{2}}\right)$


\begin{equation*}
\varphi(n)=\int_{-\infty}^{+\infty} \frac{1}{\sqrt{2 \pi \sigma^{2}}} e^{-\frac{x^{2}}{2 \sigma^{2}}} e^{i k x} d x=e^{-\frac{\sigma^{2} k^{2}}{2}} \tag{6}
\end{equation*}


Ex: - show that when the mean of the Ganssion is $\mu$

$$
\varphi(k)=e^{i k \mu-\frac{\sigma^{2}}{2} k^{2}}
$$

\begin{itemize}
  \item Colculate the c.f. of the uniform distr. $U([a, b])$ and the $\gamma$-disinibution $P(x)=\frac{\beta^{\alpha}}{\Gamma(\alpha)} x^{\alpha-1} e^{-\beta x} \quad \begin{aligned} \alpha>0 & \\ \beta>0 . & \end{aligned}$ gomma function
\end{itemize}

\end{document}