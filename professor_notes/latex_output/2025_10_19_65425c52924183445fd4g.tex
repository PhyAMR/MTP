\documentclass[10pt]{article}
\usepackage[utf8]{inputenc}
\usepackage[T1]{fontenc}
\usepackage{amsmath}
\usepackage{amsfonts}
\usepackage{amssymb}
\usepackage[version=4]{mhchem}
\usepackage{stmaryrd}
\usepackage{bbold}

\begin{document}
Models TP, 3/10/25\\
(3) $\quad \int_{-\infty}^{+\infty} e^{-\frac{a x^{2}}{2}+b x} d x=\sqrt{\frac{2 \pi}{a}} e^{\frac{b^{2}}{2 a}} \quad a>0, b \in \mathbb{C}$\\
(4) $\varphi(k)=\int e^{i k x} p(x) d x=\left\langle e^{i k x}\right\rangle$\\
(5) $\left.\quad(-i)^{n} \frac{d^{n} \varphi}{d x^{n}}\right|_{k=0}=\left\langle x^{n}\right\rangle$\\
(6) $\quad \int_{-\infty}^{+\infty} \frac{1}{\sqrt{2 \pi \sigma^{2}}} e^{-\frac{x^{2}}{2 \sigma^{2}}} e^{i n x} d x=e^{-\frac{\sigma^{2} k^{2}}{2}}$

Beconse of of. (5)

$$
\langle x\rangle=-\left.i \frac{d}{d k} e^{-\frac{\Delta^{2} k^{2}}{2}}\right|_{k=0}=0
$$

Important Gaussion iniegrals :\\
$\left\langle x^{n}\right\rangle=\left.(-i)^{n} \frac{d^{n}}{d k^{n}} e^{-\frac{\sigma^{2} k^{2}}{2}}\right|_{k=0}=0 \quad$ if $m$ is odd (by symmetry)\\
however

$$
\begin{gathered}
\left\langle x^{4}\right\rangle=\left.\frac{d^{4}}{d k^{4}} e^{-\frac{\sigma^{2} k^{2}}{2}}\right|_{k=0}=\left[\sigma^{4}\left(3-6 k^{2} \sigma^{2}+k^{4} \sigma^{4}\right) e^{-\frac{\sigma^{2} k^{2}}{2}}\right]_{k=0}= \\
=3 \sigma^{4}=\frac{1}{\sqrt{2 \pi \sigma^{2}}} \int_{-\infty}^{+\infty} c^{-\frac{x^{2}}{2 \sigma^{2}}} x^{4} d x .
\end{gathered}
$$

If we want to calculate $\left\langle x^{n}\right\rangle$, we better itart from

$$
\int_{-\infty}^{+\infty} e^{-\frac{a x^{2}}{2}} d x=\sqrt{\frac{2 \pi}{a}}
$$

We differentiaie both sides wrt a:\\
onec

$$
\int_{-\infty}^{+\infty} x^{2} e^{-\frac{a x^{2}}{2}} d x=\frac{\sqrt{2 \pi}}{a^{3 / 2}} \rightarrow\left\langle x^{2}\right\rangle
$$

twice

$$
\int_{-\infty}^{+\infty} x^{4} e^{-\frac{a x^{2}}{2}} d x=\frac{3 \sqrt{2 \pi}}{a^{5 / 2}}
$$

in times $n$ is even

$$
\int_{-\infty}^{+\infty} x^{n} e^{-\frac{a x^{2}}{2}} d x=\frac{(n-1)(n-3) \cdots 5 \cdot 3 \cdot 1 \sqrt{2 \pi}}{a^{(n+1) / 2}}
$$

From this find the expression of $\left\langle x^{n}\right\rangle$ as a function of 6 and $n$ (aven).

Multidimensional Gaussion DuTegrals

Example:

$$
\int_{-\infty}^{+\infty} d x_{1} \int_{-\infty}^{+\infty} d x_{2} e^{-\frac{3}{2}\left(x_{1}^{2}+x_{2}^{2}\right)+x_{1} x_{2}}=?
$$

Ex: write down the exponent in the form $-\frac{1}{2} \vec{x}^{\top} A \vec{x}$.\\
More generally,

$$
z(A)=\int_{\mathbb{R}^{m}} d \hat{x} e^{-\frac{1}{2} \vec{x}^{\top} A \vec{x}}
$$

Where $\vec{x}=\left(x_{1}, \ldots, x_{n}\right)$ and the matrix $A$ is diagonalizable with strictly positive eigenvalues (poritive difinite).\\
Then there exist an orthogoual matrix $O \quad\left(\Rightarrow 0 \cdot O^{\top}=O^{\top} \cdot O=11\right)$ such that we can define $\vec{y}=O \vec{x}$ and $O A O^{\top}=\Lambda$

$$
\Lambda=\left(\begin{array}{cc}
\lambda_{1} \lambda_{2} & 0 \\
0 & \ddots \\
\lambda_{m}
\end{array}\right) \quad \lambda_{i}>0 \quad i=1,2, \ldots m
$$

It follows that

$$
\vec{x}^{\top} A \vec{x}=\vec{x}^{\top} O^{\top} \wedge O \vec{x}=\vec{y}^{\top} \wedge \vec{y}
$$

$z(A)=\int d_{x}^{*} e^{-\frac{1}{2} \vec{x}^{\top} A \vec{x}}=\int d y\left\|\frac{\partial \vec{x}}{\partial \vec{y}}\right\| e^{-\frac{1}{2} \vec{y}^{\top} A \vec{y}}$\\
determinani of the Jacobion : $\left\|\frac{\partial \vec{x}}{\partial \vec{y}}\right\|=\operatorname{det}\left(O^{\top}\right)=1$ (show as an evere.)

$$
\vec{y}^{\top} \wedge \vec{y}=\sum_{i j} y_{i} \Lambda_{i j} y_{j}=\sum_{i j} y_{i} \lambda_{i} \delta_{i j} y_{j}=\sum_{i} \lambda_{i} y_{i}^{2}
$$

From this we get

$$
\begin{aligned}
& =\int_{\mathbb{R}^{m}} d^{n} y e^{-\frac{1}{2} \sum_{i} \lambda_{i} y_{i}^{2}}=\prod_{i}^{n} \int_{-\infty}^{+\infty} d y_{i} e^{-\frac{1}{2} \lambda_{i} y_{i}^{2}}=\pi_{i} \sqrt{\frac{2 \pi}{\lambda_{i}}}=\frac{(2 n)^{n / 2}}{\sqrt{\lambda_{1} \cdots \lambda_{m}}} \\
& \operatorname{det}(A)=\operatorname{det}\left(O^{\top} \Lambda O\right)=\operatorname{det}(\Lambda)(\operatorname{det} O)^{2}=\operatorname{det} \Lambda=\lambda_{1} \cdots \lambda_{\eta}
\end{aligned}
$$

Hence\\
(8) $\quad z(A)=\frac{(2 \pi)^{m / 2}}{\sqrt{\operatorname{det} A}}=\int_{\mathbb{R}^{m}} d \hat{x} \quad e^{-\frac{1}{2} \vec{x}^{\top} A \vec{x}}$

By using (8) show that

$$
\int_{-\infty}^{+\infty} d x_{1} \int_{-\infty}^{+\infty} d x_{2} e^{-\frac{3}{2}\left(x_{1}^{2}+x_{2}^{2}\right)+x_{1} x_{2}}=\frac{\pi}{\sqrt{2}}
$$

Exercife: Let $p(x, y)=\frac{\sqrt{\operatorname{det} A}}{2 \pi} e^{-\frac{1}{2}\left(a_{11} x^{2}+2 a_{12} x y+a_{22} y^{2}\right)}$ where $a_{11}, a_{22}>0$. Show that $q(x)=\int p(x, y)$ dy is still a Gaussion distribution. Find the corresponding variance of $x$. This is also true for m-dimension Gaussion verides.

We want now to calculate\\
(9) $z(A, \vec{b})=\int d^{n} x e^{-\frac{1}{2} \vec{x}^{\top} A \vec{x}+\vec{x}^{\top} \cdot \vec{b}}$\\
we use the same strategy we used before:

$$
\begin{array}{r}
\vec{\nabla}_{x}\left(-\frac{1}{2} \vec{x}^{\top} A \vec{x}+\vec{x}^{\top} \cdot \vec{b}\right)=-A \vec{x}+\vec{b}=0 \Rightarrow \vec{x}=A^{-1} \vec{b} \\
(\operatorname{det} A \neq 0)
\end{array}
$$

We introduce

$$
\begin{aligned}
\vec{y} & =\vec{x}-A^{-1} \vec{b} \\
-\frac{1}{2} \vec{x}^{\top} A \vec{x}+\vec{x}^{\top} \cdot \vec{b} & =-\frac{1}{2} \vec{y}^{\top} A \vec{y}+\frac{\vec{b}^{\top} A^{-1} \vec{b}}{2} \quad \text { (do the calculation). }
\end{aligned}
$$

Hence

$$
\begin{array}{r}
z(A, \vec{b})=\int_{\mathbb{R}^{m}} d \vec{y} e^{-\frac{1}{2} \vec{y}^{\top} A \vec{y}+\frac{\vec{b}^{\top} A^{\prime} \vec{b}}{2}}=e^{\frac{\vec{b}^{\top} A^{-1} \vec{b}}{2}} \frac{z(A, 0)}{\overline{\widehat{T}}} \\
\text { known from. Eg. 8 }
\end{array}
$$


\begin{equation*}
z(A, \vec{b})=\frac{(2 \pi)^{n / 2}}{\sqrt{\operatorname{det} A}} e^{\frac{\vec{b}^{\top} A^{-1} \vec{b}}{2}} \tag{10}
\end{equation*}


Eq. (10) allows to find the ch.f. of the multivariare Gaussion distrib.

$$
p(\vec{x})=\frac{1}{z(A, 0)} e^{-\frac{1}{2} \vec{x}^{\top} A \vec{x}}
$$

$$
\int_{\mathbb{R}^{m}} d \hat{x} p(x) e^{i \bar{k} \cdot \bar{x}}=e^{-\frac{\vec{k}^{\top} A^{-1} \vec{k}}{2}}=\varphi(k)
$$

What is the meaning of $A^{-1}$ ?\\
The definition of the ch. $f$. in the multidim. case is:

$$
\varphi(\vec{k})=\int \hat{d} x e^{i \vec{k} \cdot \vec{x}} p(\vec{x}) \quad \vec{k}=\left(k_{1}, k_{2}, \ldots, k_{m}\right)
$$

therefore we derive


\begin{equation*}
\left.(-i)^{s} \underbrace{\frac{\partial}{\partial k_{i}} \frac{\partial}{\partial k_{j}} \cdots \frac{\partial}{\partial k_{l}}}_{s \text { mony desiv. }} \varphi(\vec{k})\right|_{\vec{k}=0}=\int d^{s} x x_{i} x_{j} \cdots x_{l} p(\vec{x})=\underbrace{\left\langle x_{i} x_{j} \cdots x_{l}\right\rangle}_{\substack{S-p o i n t \text { comelation function } \\ S-\text { voriables }}} \tag{12}
\end{equation*}


Let's colculate the 2 -point conelation function for a Goussion distr.


\begin{equation*}
\left\langle x_{i} x_{j}\right\rangle=\left.(-i)^{2} \frac{\partial}{\partial k_{i}} \frac{\partial}{\partial k_{j}} e^{-\frac{\vec{k}^{\top} A^{-1} \vec{k}}{2}}\right|_{\vec{k}=0}=\left(A^{-1}\right)_{i j} \tag{13}
\end{equation*}


$A^{-7}$ is the 2 -point conelation function between a pair of Gauss. r.v. When $A^{-1}$ is a diagon. motrix, we say that the vars ore uncomel.

In the previous example: $\quad A=\left(\begin{array}{cc}3 & -1 \\ -1 & 3\end{array}\right) \quad A^{-1}=\frac{1}{8}\left(\begin{array}{ll}3 & 1 \\ 1 & 3\end{array}\right)$\\
hence

$$
\left\langle x_{1}^{2}\right\rangle=\frac{3}{8}=\left\langle x_{2}^{2}\right\rangle \quad\left\langle x_{1} x_{2}\right\rangle=\frac{1}{8}=\left\langle x_{2} x_{1}\right\rangle
$$

Notice that, because of symmetry, the s-point conel. funct. for s-variable ( $s$ is odd) is zero (the Genssion remains unchanged when $\vec{x} \rightarrow-\vec{x}$ ).

What happens when we calculate $\left\langle x_{i} x_{j} \ldots x_{e}\right\rangle$ ?\\
Should we do all the derivatives like in ey. (12)? No!\\
If the vars are Gaussion then we can use:

Wick's Heoremen. Any conelation between an even number of zero-mean Gaussion 2.v. can be written down as a sum of products of 2 -point conclation function $S\left(A^{-1}\right)$.\\
For instance:\\
$\left\langle x_{a} x_{b} x_{c} x_{d}\right\rangle=\underbrace{\left\langle x_{a} x_{b}\right\rangle}_{\left(A^{-1}\right)_{a b}} \underbrace{\left.x_{c} x_{d}\right\rangle}_{\left(A^{-1}\right)_{c d}}+\left\langle x_{a} x_{c}\right\rangle\left\langle x_{b} x_{d}\right\rangle+\left\langle x_{a} x_{d}\right\rangle\left\langle x_{b} x_{c}\right\rangle$\\
In general\\
(14)

$$
\langle\underbrace{x_{i} x_{j} \cdots x_{m} x_{m}}_{s-\text { vers }}\rangle=\sum_{p}\left(A^{-1}\right)_{i j_{p}} \cdots\left(A^{-1}\right)_{m_{p} m_{p}}
$$

where the sum is over all possible poinings of $s$ indexes, i.e. over all ways of grouping $s$ (even) indexed $s i, j \ldots, m, m$ into pairs (courting poins ever when indexes one equal).\\
Exercige: show that

$$
\begin{aligned}
& \left\langle x_{1}^{2} x_{2}^{2}\right\rangle=\frac{3}{8} \cdot \frac{3}{8}+\frac{1}{8} \cdot \frac{1}{8}+\frac{1}{8} \cdot \frac{1}{8}=\frac{11}{64} \\
& \left\langle x_{1}^{4}\right\rangle=3\left(\frac{3}{8}\right)^{2}=\left\langle x_{2}^{4}\right\rangle \quad\left\langle x_{1} x_{2}^{2}\right\rangle=
\end{aligned}
$$

J. Zimn-Justin, Quantum Field Thery and Gitical Phenomena\\
(ch. 1).

Important results obtrimed with choracteristic functions\\
If we are given the joint probability density function $p\left(x, x_{2}\right)$ and it Keppens that $p\left(x_{1}, x_{2}\right)=p_{1}\left(x_{1}\right) p_{2}\left(x_{2}\right) \quad\left(p_{1} \neq p_{2}\right)$ then the two vors are independent. Of, a top of this, $p_{1}=p_{2}$, then we soy that the two 2.v. are independent and identically distributed (ii.d.).

If we are given two z.v. that are i.i.d. con we colulde the distribution of their sum?\\
If $x_{1} \sim q(x)$ and $x_{2} \sim q(x)$ what is the distribution of $x=x_{1}+x_{2}$ ?


\end{document}