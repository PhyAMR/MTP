\documentclass[10pt]{article}
\usepackage[utf8]{inputenc}
\usepackage[T1]{fontenc}
\usepackage{amsmath}
\usepackage{amsfonts}
\usepackage{amssymb}
\usepackage[version=4]{mhchem}
\usepackage{stmaryrd}
\usepackage{bbold}

\begin{document}
\section*{Mathematical Tods}
\section*{Gaussian Integrals}
Consider the Gaussion (or Norusal) distribution (PDF):

$$
P(x)=c e^{-\frac{a x^{2}}{2}} \quad a>0
$$

$p(x)$ must be normalized on $(-\infty,+\infty)$, hence we must calculate $c$ as

$$
\int_{-\infty}^{+\infty} p(x) d x=1 \Rightarrow \quad c \int_{-\infty}^{+\infty} e^{-\frac{a x^{2}}{2}} d x=1
$$

This gives the simplest form of Gaussion initgrals:\\
(1)

$$
\int_{-\infty}^{+\infty} e^{-a \frac{x^{2}}{2}} d x=\sqrt{\frac{2 \pi}{a}}, a>0
$$

A more general Gaussion imignal is


\begin{equation*}
\int_{-\infty}^{+\infty} e^{-2 \frac{x^{2}}{2}+6 x} d x=? \tag{2}
\end{equation*}


To calculate this infegral we use a change of ver. Which one? Natice that the min of the exponent has changed. Let's find the new min:

$$
\frac{d}{d x}\left(-a \frac{x^{2}}{2}+b x\right)=-a x+b=0 \Rightarrow x=\frac{b}{a}
$$

b wal for now\\
We then insiaduce $y=x-\frac{b}{a}$\\
$-\frac{a x^{2}}{2}+b x=-\frac{a}{2}\left(y+\frac{b}{a}\right)^{2}+b\left(y+\frac{b}{a}\right)=-\frac{a}{2}\left(y^{2}+\frac{2 b y}{a}+\left(\frac{b}{a}\right)^{2}\right)+b y+\frac{b^{2}}{a}=-\frac{a y^{2}}{2}+\frac{b^{2}}{2 a}$ hence


\begin{equation*}
\int_{-\infty}^{+\infty} e^{-\frac{a x^{2}}{2}+b x} d x=\int_{-\infty}^{+\infty} e^{-\frac{a y^{2}}{2}+\frac{b^{2}}{2 a}} d y=\sqrt{\frac{2 \pi}{a}} e^{\frac{b^{2}}{2 a}} \tag{3}
\end{equation*}


here $b$ con be a complex number. Sometimes of. (3) is useful written as\\
(2)

$$
e^{\frac{b^{2}}{2 a}}=\sqrt{\frac{a}{2 \pi}} \int_{-\infty}^{+\infty} e^{-\frac{a x^{2}}{2}+b x} d x
$$

Hubbard-Stratomovich transformation\\
because it transforms a quodratic term $\left(\frac{b^{2}}{20}\right)$ into a linear one $\left(b_{x}\right)$ and this may facilitate some colulations, especially in high dimensions.

Let us colculate the integral in el. (3) when $b=i t$, t real.

$$
\begin{array}{rlrl}
\varphi(t) \equiv \int_{-\infty}^{+\infty} d x e^{i x t} \sqrt{\frac{a}{2 \pi}} e^{-\frac{a x^{2}}{2}} & \\
\varphi^{\prime}(t)=\sqrt{\frac{a}{2 \pi} i \int d x \times e^{i x t} e^{-\frac{a x^{2}}{2}}} & \leftarrow & \begin{array}{c}
\frac{d}{d t} \int f(x, t) d x=\int \frac{\partial f(x, t)}{\partial t} d x \\
\text { if } f, \partial_{t} f \text { are contimuous } \\
\mid f\left(x, t| |<A(x),\left|\partial_{t} f(x, t)\right|<B(x)\right.
\end{array} \\
& \quad \text { (A(x)dx<m, } \int 3(x) d x<\infty . \\
& \quad \text { (continuous and miformey bounded) } \\
=-\frac{i}{\sqrt{2 \pi a}} \int_{-\infty} d x e^{i x t} \frac{d}{d x} e^{-\frac{a x^{2}}{2}} & \leftarrow \frac{d}{d x} e^{-a x^{2}}=-a x e^{-a x^{2}} \\
=-\frac{t}{\sqrt{2 \pi a}} \int d x e^{i x t} e^{-\frac{a x^{2}}{2}} & \leftarrow \text { by ports } \\
= & -\frac{t}{a} \varphi(t)
\end{array}
$$

This differential equation tos solution $\varphi(t)=c e^{-\frac{t^{2}}{2 a}}$ (check this!) and $c=1$ beconse $\varphi(0)=1$.\\
Therefore

$$
\varphi(t)=e^{-\frac{t^{2}}{2 a}}=e^{\frac{b^{2}}{2 a}}
$$

\section*{Choracteristic Functions}
If we are given a PDF $P(x)$, its c. $f$. is defined as\\
(4) $\underline{\varphi(k) \equiv \int e^{i k x} p(x) d x \equiv\left\langle e^{i k x}\right\rangle \quad\left[\langle f(x)\rangle \equiv \int f(x) p(x) d x\right]}$

Notice that $\varphi(k)$ has a very wice property:\\
$\frac{d \varphi}{d k}=\int(i x) e^{i k x} p(x) d x \rightarrow-\left.i \frac{d \varphi}{d k}\right|_{k=0}=\int x p(x) d x \equiv\langle x\rangle$ and in general


\begin{equation*}
\left.(-i)^{n} \frac{d \tilde{\varphi}}{d k^{n}}\right|_{k=0}=\int x^{n} p(x) d x \equiv\left\langle x^{n}\right\rangle \quad n=1,2 \ldots \tag{5}
\end{equation*}


We will study other properties in the fellowing.\\
What is the c.f. of a Gaussion pdf? Look at op. (2), if you suls $b \rightarrow$ ik you get


\begin{equation*}
\int_{-\infty}^{+\infty} e^{-\frac{a x^{2}}{2}+b x} d x=\int_{-\infty}^{+\infty} e^{-\frac{a x^{2}}{2}+i k x}=\sqrt{\frac{2 \pi}{a}} e^{-\frac{k^{2}}{2 a}} \tag{3}
\end{equation*}


hence the c.f. of a Goussion polf $\sqrt{\frac{1}{2 \pi \sigma^{2}}} e^{-\frac{x^{2}}{2 \sigma^{2}}}\left(a=\frac{1}{\sigma^{2}}\right)$ is


\begin{equation*}
\int_{-\infty}^{+\infty} \frac{1}{\sqrt{2 \pi \sigma^{2}}} e^{-\frac{x^{2}}{2 \sigma^{2}}} e^{i k x} d x=e^{-\frac{\sigma^{2}}{2} k^{2}}=\varphi(k) \tag{6}
\end{equation*}


(4) Because of (5):

$$
\langle x\rangle=-\left.i \frac{d}{d k}\left(e^{-\frac{\sigma^{2} k^{2}}{2}}\right)\right|_{k=0}=0=\frac{1}{\sqrt{2 \pi \sigma^{2}}} \int_{-\infty}^{+\infty} e^{-\frac{x^{2}}{2 \sigma^{2}}} x d x
$$

other important Gaussion integrals:\\
$\left\langle x^{n}\right\rangle=\left.(-i)^{n} \frac{d^{n}}{d k^{n}} e^{-\frac{\sigma^{2} k^{2}}{2}}\right|_{k=0}=0=\frac{1}{\sqrt{2 \pi \sigma^{2}}} \int_{-\infty}^{+\infty} e^{-\frac{x^{2}}{2 \sigma^{2}} x^{n}} d x \quad$ if $n$ is odd\\
$\left\langle x^{2}\right\rangle=-\frac{d^{2}}{d k^{2}} e^{-\frac{\sigma^{2} k^{2}}{2}}=\sigma^{2}\left(1+\sigma^{2} k^{2}\right) e^{-\frac{\sigma^{2} k^{2}}{2}} \underset{k=0}{\rightarrow} \sigma^{2}=\frac{1}{\sqrt{2 \pi \sigma^{2}}} \int_{-\infty}^{+\infty} e^{-\frac{x^{2}}{2 \sigma^{2}}} x^{2} d x$\\
$\left\langle x^{4}\right\rangle=\frac{d^{4}}{d k^{4}} e^{-\frac{\sigma^{2} k^{2}}{2}}=\sigma^{4}\left(3-6 k^{2} \sigma^{2}+k^{4} \sigma^{4}\right) e^{-\frac{\sigma^{2} k^{2}}{2}} \underset{k=0}{\longrightarrow} 3 \sigma^{4}=\frac{1}{\sqrt{2 \pi \sigma^{2}}} \int_{-\infty}^{+\infty} e^{-\frac{x^{2}}{2 \sigma^{2}}} x^{4} d x$\\
Con we find $\left\langle x^{n}\right\rangle=$ ? when $n$ is a general even integer? Yes, ... another way:\\
From

$$
\int_{-\infty}^{+\infty} e^{-\frac{a x^{2}}{2}} d x=\sqrt{\frac{2 \pi}{a}}
$$

We differentiate both sides w.r.t. a: (con we?)

$$
\begin{array}{ll}
\int_{-\infty}^{+\infty} x^{2} e^{-\frac{a x^{2}}{2}} d x=\frac{\sqrt{2 \pi}}{a^{3 / 2}} & \text { once } \quad \text { find }\left\langle x^{2}\right\rangle \\
\int x^{4} e^{-\frac{a x^{2}}{2}} d x=\frac{3 \sqrt{2 \pi}}{a^{5 / 2}} & \text { twice } \\
\int x^{6} e^{-\frac{a x^{2}}{2}} d x=\frac{5 \cdot 3 \sqrt{2 \pi}}{a^{7 / 2}} & \text { thuice }
\end{array}
$$

$n$ even $\int_{-\infty}^{+\infty} x^{n} e^{-\frac{a x^{2}}{2}} d x=\frac{(n-1)(n-3) \cdots 5 \cdot 3 \cdot 1 \sqrt{2 \pi}}{Q^{(n+1) / 2}} \rightarrow$ find $\left\langle x^{n}\right\rangle$\\
hence another Gaussion integral

$$
\int_{-\infty}^{+\infty} x^{n} e^{-\frac{x^{2}}{2}} d x=(n-1)!!\sqrt{2 \pi}
$$

Multidimcusional Gaussian Integrals.

Example:

$$
\star \quad \int_{-\infty}^{+\infty} d x_{1} \int_{-\infty}^{+\infty} d x_{2} e^{-\frac{3}{2}\left(x_{1}^{2}+x_{2}^{2}\right)+x_{1} x_{2}}=?
$$

Ex: write down the exponent in the form $-\frac{1}{2} \vec{x}^{\top} A \vec{x}$ and find $A$.

More generally,


\begin{equation*}
z(A)=\int_{\mathbb{R}^{n}} d x^{n} e^{-\frac{1}{2} \bar{x}^{\top} A \bar{x}}=? \quad \vec{x} \equiv\left(x_{1}, x_{2} \ldots, x_{n}\right) \tag{7}
\end{equation*}


Where $\vec{x}^{\top} A \vec{x} \equiv \sum_{i j} x_{i} A_{i j} x_{j}, A$ is diagonalizable with strictly peritive eigenvolues.

There exist an outhogend matrix $O$ (namely, $O O^{\top}=0^{\top} O=11$ ) such that $\vec{y}=O \vec{x}$ and $O A O^{\top}=\Lambda$ where

$$
\Lambda=\left(\begin{array}{ccc}
\lambda_{1} & \lambda_{2} & 0 \\
0 & \ddots & \lambda_{n}
\end{array}\right) \quad \lambda_{i}>0 \quad \forall i=1,2 \ldots
$$

Therefore

$$
\begin{gathered}
\overrightarrow{x^{\top}} A \vec{x}=\vec{x}^{\top} O^{\top} \Lambda O \vec{x}=\vec{y}^{\top} \Lambda \vec{y} \\
z(A)=\int_{\mathbb{R}^{n}} d x^{n} e^{-\frac{1}{2} \vec{x}^{\top} A \vec{x}}=\int_{\mathbb{R}^{n}} d y^{n}\left\|\frac{\partial \vec{x}}{\partial \vec{y}}\right\| e^{-\frac{1}{2} \vec{y}^{\top} A \vec{y}}= \\
\text { 年 Jeterminant of } \\
\vec{y}^{\top} \Lambda \vec{y} \equiv \sum_{i j} y_{i} \Lambda_{i j} y_{j}=\sum_{i j} y_{i} \lambda_{i} \delta_{i j} y_{j}=\sum_{i} \lambda_{i} y_{i}^{2}
\end{gathered}
$$

(6)


\begin{equation*}
=\int_{\mathbb{R}^{n}} d y^{n} e^{-\frac{1}{2} \sum_{i} \lambda_{i} y_{i}^{2}}=\prod_{i}^{n} \int_{-\infty}^{+\infty} d y_{i} e^{-\frac{1}{2} \lambda_{i} y_{i}^{2}}=\prod_{j}^{n} \cdot \sqrt{\frac{2 \pi}{\lambda_{i}}}=\frac{(2 \pi)^{n / 2}}{\sqrt{\lambda_{1} \cdots \lambda_{n}}} \tag{1}
\end{equation*}


$\operatorname{det}(A)=\operatorname{det}\left(0^{\top} \Lambda 0\right)=\operatorname{det}(\Lambda)(\operatorname{det} 0)^{2}=\operatorname{det} \Lambda=\lambda_{1} \cdots \lambda_{n}$\\
(8)

$$
z(A)=\frac{(2 \pi)^{m / 2}}{\sqrt{\operatorname{det} A}}
$$

Show that

$$
\int_{-\infty}^{+\infty} d x_{1} \int_{-\infty}^{+\infty} d x_{2} e^{-\frac{3}{2}\left(x_{1}^{2}+x_{2}^{2}\right)+x_{1} x_{2}}=\frac{\pi}{\sqrt{2}}
$$

Exercije:\\
Let $p(x, y)=\frac{\sqrt{\operatorname{det} A}}{2 \pi} e^{-\frac{1}{2}\left(a_{11} x^{2}+2 a_{12} x y+a_{22} y^{2}\right)}$, where $A=\left(\begin{array}{ll}a_{11} & a_{12} \\ a_{21} & a_{22}\end{array}\right) ; a_{11}, a_{22}>0, a_{12}=a_{21}$ Show that $\int p(x, y) d y$ is still a Gaussion polf. Find the cornesponding veriance of the 2.V. $X$.

We wish to colculate


\begin{equation*}
z(A, \vec{b})=\int_{\mathbb{R}^{n}} d^{n} x e^{-\frac{1}{2} \bar{x}^{\top} A \bar{x}+\vec{x}^{\top} \cdot \vec{b}} \tag{9}
\end{equation*}


we will follow the strotegy we used before, by shifting the minimum:

$$
\begin{gathered}
\vec{\nabla}_{x}\left(-\frac{1}{2} \vec{x}^{\top} A \vec{x}+\vec{x}^{\top} \cdot \vec{b}\right)=-A \vec{x}+\vec{b}=0 \quad \Rightarrow \quad \vec{x}=A^{-1} \vec{b} \quad\binom{\text { det } A \neq 0}{A \text { symm. }} \\
\vec{y}=\vec{x}-A^{-1} b \\
-\frac{1}{2} \vec{x}^{\top} A \vec{x}+\vec{x}^{\top} \vec{b}=-\frac{1}{2} \vec{y}^{\top} A \vec{y}+\frac{\vec{b}^{\top} A^{-1}}{2} \quad \text { (do the calcul.) }
\end{gathered}
$$

Hence

$$
z(A, \overrightarrow{5})=\int_{\mathbb{R}^{n}} d \vec{y} e^{-\frac{1}{2} \vec{y}^{\top} 4 \vec{y}+\frac{\vec{b}^{\top} A^{-1} \vec{b}}{2}}=z(A, \overrightarrow{0}) e^{\frac{\vec{b}^{\top} A^{-1} \vec{b}}{2}}
$$

(10)

$$
z(A, 5)=\frac{(2 \pi)^{M / 2}}{\sqrt{\operatorname{det} A}} e^{\frac{5^{\top} A^{-1} \Gamma}{2}}
$$

$$
\vec{b}^{\top} A^{-1} \vec{b} \equiv \sum_{i j=1}^{n} b_{i}\left(A^{-1}\right)_{i j} b_{j}
$$

Eq. (10) allows to find the charact. function of a multivariate Gaussion pdf


\begin{equation*}
p(\vec{x})=\frac{1}{z(\lambda, \bar{\sigma})} e^{-\frac{1}{2} \vec{x}^{\top} \Delta \vec{x}} \underset{\substack{\vec{\hbar} \rightarrow i \vec{k} \\ \text { and } \varphi!\cdot(\mathbb{R})}}{\longrightarrow} \int_{\mathbb{R}^{n}} d \vec{x} p(\vec{x}) e^{i \vec{k} \cdot \vec{x}}=e^{-\frac{\vec{k}^{\top} A^{-1} \vec{k}}{2}}=\varphi(\vec{k}) \tag{11}
\end{equation*}


What is the meaning of $A^{-1}$ ? Let's go back to $p$. (5) (in $\operatorname{dim} n$ ) $n$-dim. char. function

$$
\left.(-i)^{s} \underbrace{\frac{\partial}{\partial k_{i}} \frac{\partial}{\partial k_{j}} \cdots \frac{\partial}{\partial k_{e}}}_{\text {(which many be qual) }} \varphi(\vec{E})\right|_{\vec{k}=0}=\int_{\mathbb{R}} d_{x}^{s} x_{i} x_{j} \cdots x_{e} p(\vec{x})=\underbrace{\left\langle x_{i} x_{j} \cdots x_{e}\right\rangle}_{\nearrow}
$$

Let's calculate the 2 -point convelation for the Gaussion pof\\
(13) $\left\langle x_{i} x_{j}\right\rangle=\left.(-i)^{2} \frac{\partial}{\partial k_{i}} \frac{\partial}{\partial k_{j}} e^{-\frac{\vec{k}^{\top} A^{-1} \vec{k}}{2}}\right|_{\vec{k}=0}=\left(A^{-1}\right)_{i j}$\\
$A^{-1}$ is the 2 -point covelation funtion between a pair of Gouss. I. v. When $A^{-1}$ is a digonal matrix, we say that the vers are uncovelated. In the previous example:

$$
A^{-1}=\frac{1}{8}\left(\begin{array}{ll}
3 & 1 \\
1 & 3
\end{array}\right) \quad \text { hence } \quad\left\langle x_{1}^{2}\right\rangle=\frac{3}{8}=\left\langle x_{2}^{2}\right\rangle,\left\langle x_{1} x_{2}\right\rangle=\frac{1}{8}=\left\langle x_{2} x_{1}\right\rangle
$$


\end{document}