\lecture{1}{Gaussian Integrals}{2025-09-30}
\pagelayout{margin}
% --- Start writing here ---
Consider the Gaussion distribution (PDF) :
\begin{DispWithArrows}[format=c, displaystyle]
p(x)=c e^{-\frac{a x^{2}}{2}} \quad a>0
\end{DispWithArrows}
$p$ must be nonualized on $(-\infty,+\infty)$ :
\begin{DispWithArrows}[format=c, displaystyle]
\int_{-\infty}^{+} p(x) d x=1 \quad \Rightarrow \quad c \int_{-\infty}^{+\infty} e^{-\frac{a x^{2}}{2}} d x=1
\end{DispWithArrows}
The simplest Gaussion integral:
\begin{DispWithArrows}[format=c, displaystyle]
\int_{-\infty}^{+\infty} e^{-\frac{a x^{2}}{2}} d x=\sqrt{\frac{2 \pi}{a}}
\end{DispWithArrows}
A more general Goussion integral is
\begin{DispWithArrows}[format=c, displaystyle]
\int_{-\infty}^{+\infty} e^{-\frac{a x^{2}}{2}+b x} d x=?
\end{DispWithArrows}
To edcalate this integral we use a charge of vars. The mox of the exponent has charged. Let's find it:
\begin{DispWithArrows}[format=c, displaystyle]
\frac{d}{d x}\left(-\frac{a x^{2}}{2}+b x\right)=-a x+b=0 \Rightarrow x=\frac{b}{a} \quad \text { (b eal) }
\end{DispWithArrows}
We introduce $y=x-\frac{b}{a}$
Verify
$-\frac{a x^{2}}{2}+b x=-\frac{a}{2}\left(y+\frac{b}{a}\right)^{2}+b\left(y+\frac{b}{a}\right)=\cdots=-\frac{a y^{2}}{2}+\frac{b^{2}}{2 a}$
\begin{DispWithArrows}[format=c, displaystyle]
\int_{-\infty}^{+} e^{-\frac{a x^{2}}{2}+b x} d x=\int_{-\infty}^{+\infty} e^{-\frac{a y^{2}}{2}+\frac{b^{2}}{2 a} d y} \xrightarrow{\text { (1) }} \sqrt{\frac{2 \pi}{a}} e^{\frac{b^{2}}{2 a}} \quad \begin{aligned} & a>0 \\ & b \in \mathbb{C}\end{aligned}
\end{DispWithArrows}
b con also be a complex number.

Let's calculate the integral in (3) when $b=i t \quad t \in \mathbb{R}$. Lat's define
\begin{DispWithArrows}[format=c, displaystyle]
\varphi(t)=\int_{-\infty}^{+\infty} d x e^{i x t} \sqrt{\frac{a}{2 \pi}} e^{-\frac{a x^{2}}{2}}
\end{DispWithArrows}
Note:$\frac{d}{d t} \int f(x, t) d x=\int \frac{\partial f(x, t)}{\partial t} d x$ if $f, \partial_{t} f$ are continuous and $|f(x, t)|<A(x) \quad,\left|\partial_{t} f\right|<B(x), \quad \int A(x) d x<\infty \quad \int B(x) d x<\infty$ ( $f$ is continuous and unifoumly bounded).
\begin{DispWithArrows}[format=rL]
\varphi^{\prime}(t)&=\sqrt{\frac{a}{2 \pi} i} \int d x \times e^{i x t} e^{-\frac{a x^{2}}{2}} \\
&=-\frac{i}{\sqrt{2 \pi a}} \int d x e^{i x t} \frac{d}{d x} e^{-\frac{a x^{2}}{2}} \quad \frac{d}{d x} e^{-a x^{2}}=-a x e^{-a x^{2}} \\
&=-\frac{t}{\sqrt{2 \pi a}} \int d x e^{i x t} e^{-\frac{a x^{2}}{2}} \quad(\text { by posts }) \\
&=-\frac{t}{a} \varphi(t) \\
\Rightarrow \quad \varphi^{\prime}&=-\frac{t}{a} \varphi
\end{DispWithArrows}
This obifferential $\varphi$. is linear and the solution is $\varphi(t)=c e^{-\frac{t^{2}}{2 a}}$ (check this out!). It must be $\varphi(0)=1 \Rightarrow c=1$
\begin{DispWithArrows}[format=rL]
\varphi(t)=e^{-\frac{t^{2}}{2 a}}=e^{\frac{b^{2}}{2 a}} \\b=i t
\end{DispWithArrows}
\section*{Characteristic Functions}
If we are given a PDF $p(x)$, its char. func. is defined as
\begin{DispWithArrows}[format=c, displaystyle]
\varphi(k)=\int e^{i k x} p(x) d x \equiv\left\langle e^{i k x}\right\rangle_{p}
\end{DispWithArrows}
In general
\begin{DispWithArrows}[format=c, displaystyle]
\langle f\rangle_{p} \equiv \int f(x) p(x) d x
\end{DispWithArrows}
$\varphi(x)$ has a very nice property:
\begin{DispWithArrows}[format=c, displaystyle]
\frac{d \varphi}{d k}=\int(i x) e^{i k x} p(x) d x \rightarrow-\left.i \frac{d \varphi}{d k}\right|_{k=0}=\int x p(x) d x=\langle x\rangle
\end{DispWithArrows}
and in general:
\begin{DispWithArrows}[format=c, displaystyle]
\left.(-i)^{n} \frac{d^{n} \varphi}{d k^{n}}\right|_{k=0}=\int x^{n} p(x) d x=\left\langle x^{n}\right\rangle
\end{DispWithArrows}
wth moment of $P$
$n=1,2, \cdots$

What is the c.f. Of the Gaussion distribution? Substitute $b \rightarrow$ itc in e. (2), then you get
\begin{DispWithArrows}[format=c, displaystyle]
\int_{-\infty}^{+\infty} e^{-\frac{a x^{2}}{2}+b x} d x=\int e^{-\frac{a x^{2}}{2}+i k x}=\sqrt{\frac{2 \pi}{a}} e^{-\frac{k^{2}}{2 a}}
\end{DispWithArrows}
hence the c.f. of the Goussion is $\left(\sqrt{\frac{1}{2 \pi \sigma^{2}}} e^{-\frac{x^{2}}{2 \sigma^{2}}}, a=\frac{1}{\sigma^{2}}\right)$
\begin{DispWithArrows}[format=c, displaystyle]
\varphi(n)=\int_{-\infty}^{+\infty} \frac{1}{\sqrt{2 \pi \sigma^{2}}} e^{-\frac{x^{2}}{2 \sigma^{2}}} e^{i k x} d x=e^{-\frac{\sigma^{2} k^{2}}{2}}
\end{DispWithArrows}
Ex: - show that when the mean of the Ganssion is $\mu$
\begin{DispWithArrows}[format=c, displaystyle]
\varphi(k)=e^{i k \mu-\frac{\sigma^{2}}{2} k^{2}}
\end{DispWithArrows}
\begin{itemize}
  \item Colculate the c.f. of the uniform distr. $U([a, b])$ and the $\gamma$-disinibution $P(x)=\frac{\beta^{\alpha}}{\Gamma(\alpha)} x^{\alpha-1} e^{-\beta x} \quad \begin{aligned} \alpha>0 & \\ \beta>0 . & \end{aligned}$ gomma function
\end{itemize}
