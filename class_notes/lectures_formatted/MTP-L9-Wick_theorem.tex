% !TeX encoding = UTF-8
% Lecture file created by newnote
% Class: Models of Theoretical Physics
% Professor: Azaele Sandro
% Date: 2025-10-17
\lecture{9}{Wick's theorem}{2025-10-17}
\pagelayout{margin}
% --- Start writing here ---

\section{Wick's theorem}
Note that, because of symmetry, the s-point correlation for $s$ any odd integer
is zero (the Gaussian remains unchanged if $\vec{x} \rightarrow-\vec{x}$). 

What happens when we want to calculate
$\left\langle x_{i} x_{j} \cdots x_{l}\right\rangle$? Should we always do all
the derivatives as in eq. (12)? No!
If the vars are Gaussian we can use

\subsection*{Wick's theorem}
Any correlation between an even number of Gaussian r.v. can be written down as
a sum of products of 2-point correlation functions $\left(A^{-1}\right)$.
For instance
\begin{DispWithArrows}[displaystyle, format=ll]
  \begin{aligned}
    \left\langle x_{a} x_{b} x_{c} x_{d}\right\rangle= & \left\langle x_{a} x_{b}\right\rangle\left\langle x_{c} x_{d}\right\rangle+\left\langle x_{a} x_{c}\right\rangle\left\langle x_{b} x_{d}\right\rangle+\left\langle x_{a} x_{d}\right\rangle\left\langle x_{b} x_{c}\right\rangle \\
    & \left(A^{-1}\right)_{a b} \quad \left(A^{-1}\right)_{c d} \quad \cdots \quad \text { (indexes may be equal) }
  \end{aligned}
\end{DispWithArrows}
Exercise: from the previous case, show that (no calculations!)
\begin{DispWithArrows}[displaystyle, format=ll]
  \begin{aligned}
    & \left\langle x_{1}^{2} x_{2}^{2}\right\rangle=\left\langle x_1 x_1\right\rangle \left\langle x_2 x_2\right\rangle + 2\left\langle x_1 x_2\right\rangle^2=\frac{3}{8} \cdot \frac{3}{8}+2\left(\frac{1}{8}\right)^2=\frac{11}{64} \\
    & \left\langle x_{1}^{4}\right\rangle=3\left\langle x_1^2\right\rangle^2=3\left(\frac{3}{8}\right)^{2}=\frac{27}{64}
  \end{aligned}
\end{DispWithArrows}
In general:
\begin{DispWithArrows}[displaystyle, format=c]
  \langle\underbrace{x_{i} x_{j} \ldots x_{n} x_{m}}_{s \text { vars }}\rangle=\sum_{p}\left(A^{-1}\right)_{i_{p} j_{p}} \cdots\left(A^{-1}\right)_{n_{p} m_{p}}
\end{DispWithArrows}
Where the sum runs over all pairings of $s$ indexes, i.e. over all ways of
grouping $s$ (even) indexes $i, j \ldots, n, m$ into pairs (counting the pair
even when indexes are equal; order in the pairs is not important). If there are
$s$ vars then the possible pairings are $(s-1)!!=(s-1)(s-3) \ldots 3 \cdot 1$.

\subsection*{Further important results obtained with characteristic functions}
(1) If you are given two independent and identically distributed (iid) random
variables, how do you calculate the pdf of their sum? What is its c.f.?
We assume they are real with pdf $q(x)$:
\begin{DispWithArrows}[displaystyle, format=c]
  $x=x_{1}+x_{2} \quad x_{1}, x_{2} \sim q(x)$
\end{DispWithArrows}
When we draw $x_{1}$ and $x_{2}$ from $q(x)$, many different outcomes can give
you the same sum $x$, these have to be added up with the corresponding
probability, hence
\begin{DispWithArrows}[displaystyle, format=c]
  $p(x)=\int \delta\left(x-\left(x_{1}+x_{2}\right)\right) p\left(x_{1}, x_{2}\right) d x_{1} d x_{2} \equiv\left\langle\delta\left(x-x_{1}-x_{2}\right)\right\rangle$
\end{DispWithArrows}
we select all possible $x_{1}, x_{2}$ s.t. their sum is $x$.
i.i.d.
\begin{DispWithArrows}[displaystyle, format=ll]
  \begin{aligned}
    & =\int \delta\left(x-x_{1}-x_{2}\right) q\left(x_{1}\right) q\left(x_{2}\right) d x_{1} d x_{2} \\
    & =\int q(x-y) q(y) d y \quad \text { it's a convolution. }
  \end{aligned}
\end{DispWithArrows}
What is the c.f. of $p(x)$ if the c.f. of $q(x)$ is $\varphi_{1}(k)$?
\begin{DispWithArrows}[displaystyle, format=ll]
  \begin{aligned}
    \varphi(k) & \equiv\left\langle e^{i k x}\right\rangle=\int e^{i k x} p(x) d x=\int d x e^{i k x} \delta\left(x-\left(x_{1}+x_{2}\right)\right) q\left(x_{1}\right) q\left(x_{2}\right) d x_{1} d x_{2} \\
    & =\int e^{i k\left(x_{1}+x_{2}\right)} q\left(x_{1}\right) q\left(x_{2}\right) d x_{1} d x_{2}=\left[\varphi_{1}(k)\right]^{2}
  \end{aligned}
\end{DispWithArrows}
What happens if the r.v. are independent but not identically distributed?
(2) The (weak) law of large numbers (convergence in distribution)
If we are now given $n$ iid r.v. whose pdf is $q(x)$ with c.f. $\varphi_{1}(k)$,
what happens to $X=\frac{1}{n} \sum_{i} x_{i}$ as $n \rightarrow \infty$?
Let's assume that the mean of $x_{i}$ is
$\mu \quad\left(\mu=\int x q(x) d x<\infty\right)$. (See Grimmett & Stirzaker,
p. 193, Prob. and Random Processes). proof:
Let $\varphi_{n}(k)$ be the c.f. of the average of the rand. variables
\begin{DispWithArrows}[displaystyle, format=c]
  \varphi_{n}(k) \equiv\left\langle e^{i k X}\right\rangle=\left\langle e^{i k \frac{1}{n} \sum_{i} x_{i}}\right\rangle=\int e^{\frac{i k}{n} \sum_{i} x_{i}} q\left(x_{1}\right) \cdots q\left(x_{n}\right) d x_{1} \cdots d x_{n}
\end{DispWithArrows}
\begin{DispWithArrows}[displaystyle, format=c]
  =\left(\int e^{i \frac{k}{n} x_{1}} q\left(x_{1}\right) d x_{1}\right) \cdots\left(\int e^{i k \frac{x_{n}}{n}} q\left(x_{n}\right) d x_{n}\right)=\left(\varphi_{1}\left(\frac{k}{n}\right)\right)^{n}
\end{DispWithArrows}
Also, Taylor
\begin{DispWithArrows}[displaystyle, format=c]
  \varphi_{1}\left(\frac{k}{n}\right) = 1+\frac{i k}{n}\langle x\rangle+O\left(\frac{1}{n^2}\right) \text { as } n \rightarrow \infty
\end{DispWithArrows}
From (17)
\begin{DispWithArrows}[displaystyle, format=c]
  \left(1+\frac{i k}{n}\langle x\rangle+\ldots\right)^{n} \xrightarrow[n \rightarrow \infty]{} e^{i \mu k}=\int \delta(x-\mu) e^{i k x} d x
\end{DispWithArrows}
Here we have only proved that we have a much stronger result:
\begin{DispWithArrows}[displaystyle, format=c]
  \frac{1}{n} \sum_{i=1}^{n} x_{i} \rightarrow \mu
\end{DispWithArrows}
The strong law of large numbers states: Let $x_{1} \ldots x_{n}$ be a sequence
of i.i.d. r.v. each with finite mean $\mu$. Then the finite (empirical) average
approaches $\mu$ as $n \rightarrow \infty$. (Grimmett, p. 329)
This law tells us that, for large $n$, the sum $\sum_{i} x_{i}$ is approximately
$n \mu$. Of course there will be fluctuations around $n \mu$. A natural question
is then what can we say about $\sum_{i=1}^{n} x_{i}-\mu n$?
There is an extraordinary answer to this question, which is valid whenever
$x_{i}$ have finite variance:
a) $\sum_{i=1}^{n} x_{i}-\mu n$ is about as big as $\sqrt{n}$
b) The distribution of $\frac{\sum_{i=1}^{n} x_{i}-\mu n}{\sqrt{n}}$ approaches a
Gaussian pdf as $n \rightarrow \infty$ IRRESPECTIVE of the pdf of $x_{i}$.

Claims a) and b) are the core meaning of the

\subsection*{Central Limit Theorem}
Let $x_{1} \ldots x_{n}$ be a sequence of i.i.d. r.v. with finite mean $\mu$ and
finite (non-zero) variance $\sigma^{2}$. Then the p.d.f. of
\begin{DispWithArrows}[displaystyle, format=c]
  $Y_{n}=\frac{\sum_{i=1}^{n} x_{i}-\mu n}{\sqrt{n} \sigma} \xrightarrow[n \rightarrow \infty]{\text{conv. in distrib.}} N(0,1)$
\end{DispWithArrows}
Obs:
$\left\langle Y_{n}\right\rangle=\frac{1}{\sqrt{n} \sigma}\left(\sum_{i}\left\langle x_{i}\right\rangle-\mu n\right)=0$
\begin{DispWithArrows}[displaystyle, format=c]
  \operatorname{Var}\left(Y_{n}\right)=\frac{1}{n \sigma^{2}} \operatorname{Var}\left(\sum_{i=1}^{n} x_{i}-\mu n\right)=\frac{1}{n \sigma^{2}} \operatorname{Var}\left(\sum_{i=1}^{n} x_{i}\right)=\frac{\sum_{i} \operatorname{Var}\left(x_{i}\right)}{n \sigma^{2}}=\frac{n \sigma^{2}}{n \sigma^{2}}=1
\end{DispWithArrows}
(please, go throughly through all the steps and revise the properties of
var(…)).
This means that $Y_{n}$ has a center (0) and a "width" that does not change as
$n$ varies.
proof:
Let's assume that each r.v. has a p.d.f. $q(x)$ with c.f. $\varphi_{1}(k)$,
$\varphi_{n}(k)$ is the c.f. of $Y_{n}$:
\begin{DispWithArrows}[displaystyle, format=c]
  \varphi_{n}(k)=\left\langle e^{i k Y_{n}}\right\rangle=\int e^{i k \frac{\sum_{i} x_{i}-\mu n}{\sqrt{n} \sigma}} q\left(x_{1}\right) \cdots q\left(x_{n}\right) d x_{1} \cdots d x_{n}=
\end{DispWithArrows}
\begin{DispWithArrows}[displaystyle, format=c]
  =e^{-\frac{i k \mu \sqrt{n}}{\sigma}}\left(\int e^{\frac{i k x}{\sqrt{n} \sigma}} q(x) d x\right)^{n} = e^{-\frac{i k \mu \sqrt{n}}{\sigma}}\left(\varphi_{1}\left(\frac{k}{\sqrt{n} \sigma}\right)\right)^{n}
\end{DispWithArrows}
As in the previous theorem we can expand $\varphi_{1}$ as $n \rightarrow \infty$
\begin{DispWithArrows}[displaystyle, format=c]
  \varphi_{1}\left(\frac{k}{\sqrt{n} \sigma}\right) = 1+\frac{i k}{\sqrt{n} \sigma}\langle x\rangle-\frac{k^{2}}{2 n \sigma^{2}}\left\langle x^{2}\right\rangle+O\left(n^{-3 / 2}ight) = e^{\frac{i k \mu}{\sqrt{n}\sigma}-\frac{k^{2}}{2n}}
\end{DispWithArrows}
from (20)
\begin{DispWithArrows}[displaystyle, format=c]
  \varphi_{n}(k)=e^{-\frac{i k \mu}{\sigma} \sqrt{n}} e^{\frac{i k \mu}{\sigma} \sqrt{n}-\frac{k^{2}}{2}}=e^{-\frac{k^{2}}{2}}
\end{DispWithArrows}
As we have shown in eq. (6), this is the c.f. of
$p(x)=\frac{1}{\sqrt{2 \pi}} e^{-\frac{x^{2}}{2}} \equiv N(0,1)$. Show 1)
$\sum_{i=1}^{n} x_{i} \sim N\left(n \mu, n \sigma^{2}\right)$; 2)
$\frac{1}{n} \sum_{i=1}^{n} x_{i} \sim N\left(\mu, \frac{\sigma^{2}}{n}\right)